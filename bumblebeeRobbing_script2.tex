% Options for packages loaded elsewhere
\PassOptionsToPackage{unicode}{hyperref}
\PassOptionsToPackage{hyphens}{url}
%
\documentclass[
]{article}
\usepackage{amsmath,amssymb}
\usepackage{iftex}
\ifPDFTeX
  \usepackage[T1]{fontenc}
  \usepackage[utf8]{inputenc}
  \usepackage{textcomp} % provide euro and other symbols
\else % if luatex or xetex
  \usepackage{unicode-math} % this also loads fontspec
  \defaultfontfeatures{Scale=MatchLowercase}
  \defaultfontfeatures[\rmfamily]{Ligatures=TeX,Scale=1}
\fi
\usepackage{lmodern}
\ifPDFTeX\else
  % xetex/luatex font selection
\fi
% Use upquote if available, for straight quotes in verbatim environments
\IfFileExists{upquote.sty}{\usepackage{upquote}}{}
\IfFileExists{microtype.sty}{% use microtype if available
  \usepackage[]{microtype}
  \UseMicrotypeSet[protrusion]{basicmath} % disable protrusion for tt fonts
}{}
\makeatletter
\@ifundefined{KOMAClassName}{% if non-KOMA class
  \IfFileExists{parskip.sty}{%
    \usepackage{parskip}
  }{% else
    \setlength{\parindent}{0pt}
    \setlength{\parskip}{6pt plus 2pt minus 1pt}}
}{% if KOMA class
  \KOMAoptions{parskip=half}}
\makeatother
\usepackage{xcolor}
\usepackage[margin=1in]{geometry}
\usepackage{color}
\usepackage{fancyvrb}
\newcommand{\VerbBar}{|}
\newcommand{\VERB}{\Verb[commandchars=\\\{\}]}
\DefineVerbatimEnvironment{Highlighting}{Verbatim}{commandchars=\\\{\}}
% Add ',fontsize=\small' for more characters per line
\usepackage{framed}
\definecolor{shadecolor}{RGB}{248,248,248}
\newenvironment{Shaded}{\begin{snugshade}}{\end{snugshade}}
\newcommand{\AlertTok}[1]{\textcolor[rgb]{0.94,0.16,0.16}{#1}}
\newcommand{\AnnotationTok}[1]{\textcolor[rgb]{0.56,0.35,0.01}{\textbf{\textit{#1}}}}
\newcommand{\AttributeTok}[1]{\textcolor[rgb]{0.13,0.29,0.53}{#1}}
\newcommand{\BaseNTok}[1]{\textcolor[rgb]{0.00,0.00,0.81}{#1}}
\newcommand{\BuiltInTok}[1]{#1}
\newcommand{\CharTok}[1]{\textcolor[rgb]{0.31,0.60,0.02}{#1}}
\newcommand{\CommentTok}[1]{\textcolor[rgb]{0.56,0.35,0.01}{\textit{#1}}}
\newcommand{\CommentVarTok}[1]{\textcolor[rgb]{0.56,0.35,0.01}{\textbf{\textit{#1}}}}
\newcommand{\ConstantTok}[1]{\textcolor[rgb]{0.56,0.35,0.01}{#1}}
\newcommand{\ControlFlowTok}[1]{\textcolor[rgb]{0.13,0.29,0.53}{\textbf{#1}}}
\newcommand{\DataTypeTok}[1]{\textcolor[rgb]{0.13,0.29,0.53}{#1}}
\newcommand{\DecValTok}[1]{\textcolor[rgb]{0.00,0.00,0.81}{#1}}
\newcommand{\DocumentationTok}[1]{\textcolor[rgb]{0.56,0.35,0.01}{\textbf{\textit{#1}}}}
\newcommand{\ErrorTok}[1]{\textcolor[rgb]{0.64,0.00,0.00}{\textbf{#1}}}
\newcommand{\ExtensionTok}[1]{#1}
\newcommand{\FloatTok}[1]{\textcolor[rgb]{0.00,0.00,0.81}{#1}}
\newcommand{\FunctionTok}[1]{\textcolor[rgb]{0.13,0.29,0.53}{\textbf{#1}}}
\newcommand{\ImportTok}[1]{#1}
\newcommand{\InformationTok}[1]{\textcolor[rgb]{0.56,0.35,0.01}{\textbf{\textit{#1}}}}
\newcommand{\KeywordTok}[1]{\textcolor[rgb]{0.13,0.29,0.53}{\textbf{#1}}}
\newcommand{\NormalTok}[1]{#1}
\newcommand{\OperatorTok}[1]{\textcolor[rgb]{0.81,0.36,0.00}{\textbf{#1}}}
\newcommand{\OtherTok}[1]{\textcolor[rgb]{0.56,0.35,0.01}{#1}}
\newcommand{\PreprocessorTok}[1]{\textcolor[rgb]{0.56,0.35,0.01}{\textit{#1}}}
\newcommand{\RegionMarkerTok}[1]{#1}
\newcommand{\SpecialCharTok}[1]{\textcolor[rgb]{0.81,0.36,0.00}{\textbf{#1}}}
\newcommand{\SpecialStringTok}[1]{\textcolor[rgb]{0.31,0.60,0.02}{#1}}
\newcommand{\StringTok}[1]{\textcolor[rgb]{0.31,0.60,0.02}{#1}}
\newcommand{\VariableTok}[1]{\textcolor[rgb]{0.00,0.00,0.00}{#1}}
\newcommand{\VerbatimStringTok}[1]{\textcolor[rgb]{0.31,0.60,0.02}{#1}}
\newcommand{\WarningTok}[1]{\textcolor[rgb]{0.56,0.35,0.01}{\textbf{\textit{#1}}}}
\usepackage{graphicx}
\makeatletter
\newsavebox\pandoc@box
\newcommand*\pandocbounded[1]{% scales image to fit in text height/width
  \sbox\pandoc@box{#1}%
  \Gscale@div\@tempa{\textheight}{\dimexpr\ht\pandoc@box+\dp\pandoc@box\relax}%
  \Gscale@div\@tempb{\linewidth}{\wd\pandoc@box}%
  \ifdim\@tempb\p@<\@tempa\p@\let\@tempa\@tempb\fi% select the smaller of both
  \ifdim\@tempa\p@<\p@\scalebox{\@tempa}{\usebox\pandoc@box}%
  \else\usebox{\pandoc@box}%
  \fi%
}
% Set default figure placement to htbp
\def\fps@figure{htbp}
\makeatother
\setlength{\emergencystretch}{3em} % prevent overfull lines
\providecommand{\tightlist}{%
  \setlength{\itemsep}{0pt}\setlength{\parskip}{0pt}}
\setcounter{secnumdepth}{-\maxdimen} % remove section numbering
\usepackage{bookmark}
\IfFileExists{xurl.sty}{\usepackage{xurl}}{} % add URL line breaks if available
\urlstyle{same}
\hypersetup{
  pdftitle={Bumblebee2},
  pdfauthor={Daniel},
  hidelinks,
  pdfcreator={LaTeX via pandoc}}

\title{Bumblebee2}
\author{Daniel}
\date{2025-06-09}

\begin{document}
\maketitle

Behavioral assays done on bumblebees at Rocky Mountains Biology
Laboratory (RMBL) field season 2023. Observations were carried out by
Nick Dabagia and Daniel Souto for Corydalis and Mertensia species.
Y-Tube assays carried out by Oriana Gutierrez in 2024.

Methods:

Refer to Souto-Vilarós et al.~`Yeast volatiles promote larceny in bumble
bee behavior.'

Data analysis:

The dataset is arranged as date, plant species, bee individual, bee
species, whether or not a choice was made, stalk number, treatment,
whether or not the bee tried to legitimately visit the flower, whether
it robbed the flower, time to rob, time feeding, flower number (total
for that bout) and additional notes.

Note that not all packages are used in the current analysis

We need to wrangle the data a bit. Namely: Filter the data to include
only cases where a flower was robbed (column CHOICE, ``Yes''), Remove B.
mixtus (we never got enough individuals), Remove an outlier which was
inactive on a flwoer over 3 minutes. Make sure all data is numeric and
there are no missing data.

This final data set includes: 57 total successful robber trials (out of
116 bee trials transcribed) 42 Corydalis 2° robbers (22 bifarius, 16
flavifrons, 4 mixtus) and 15 Mertensia 2° robbers (all flavifrons)

\begin{Shaded}
\begin{Highlighting}[]
\NormalTok{robbed\_flowers }\OtherTok{\textless{}{-}}\NormalTok{ visitation }\SpecialCharTok{\%\textgreater{}\%}
  \FunctionTok{filter}\NormalTok{(robbed }\SpecialCharTok{==} \StringTok{"yes"}\NormalTok{) }\SpecialCharTok{\%\textgreater{}\%} \CommentTok{\#note that this will reduce total number of bees which made a choice since not all of them actually \textquotesingle{}robbed\textquotesingle{}}
  \FunctionTok{filter}\NormalTok{(species }\SpecialCharTok{!=} \StringTok{"mixtus"}\NormalTok{)}\SpecialCharTok{\%\textgreater{}\%}
  \FunctionTok{filter}\NormalTok{(timetorob }\SpecialCharTok{!=}\FloatTok{154.7}\NormalTok{) }

\NormalTok{robbed\_flowers }\OtherTok{\textless{}{-}}\NormalTok{ robbed\_flowers }\SpecialCharTok{\%\textgreater{}\%}
  \FunctionTok{mutate}\NormalTok{(}
    \AttributeTok{timetorob =} \FunctionTok{as.numeric}\NormalTok{(timetorob),}
    \AttributeTok{timefeeding =} \FunctionTok{as.numeric}\NormalTok{(timefeeding)}
\NormalTok{  ) }\SpecialCharTok{\%\textgreater{}\%}
  \FunctionTok{filter}\NormalTok{(}\SpecialCharTok{!}\FunctionTok{is.na}\NormalTok{(timetorob) }\SpecialCharTok{\&} \SpecialCharTok{!}\FunctionTok{is.na}\NormalTok{(timefeeding))}

\NormalTok{robbed\_flowers }\OtherTok{\textless{}{-}}\NormalTok{ robbed\_flowers }\SpecialCharTok{\%\textgreater{}\%}
  \FunctionTok{mutate}\NormalTok{(}\AttributeTok{trytolegit\_binary =} \FunctionTok{ifelse}\NormalTok{(trytolegit }\SpecialCharTok{==} \StringTok{"yes"}\NormalTok{, }\DecValTok{1}\NormalTok{, }\DecValTok{0}\NormalTok{))}

\CommentTok{\#How many bees per bee species per flower species in the original and the filtered datasets?}

\NormalTok{visitation }\SpecialCharTok{\%\textgreater{}\%} \FunctionTok{group\_by}\NormalTok{(sample, plantsp, species) }\SpecialCharTok{\%\textgreater{}\%} \FunctionTok{count}\NormalTok{()}
\end{Highlighting}
\end{Shaded}

\begin{verbatim}
## # A tibble: 120 x 4
## # Groups:   sample, plantsp, species [120]
##    sample     plantsp   species      n
##    <chr>      <chr>     <chr>    <int>
##  1 CORYBIF001 corydalis bifarius     5
##  2 CORYBIF002 corydalis bifarius     1
##  3 CORYBIF003 corydalis bifarius     2
##  4 CORYBIF004 corydalis bifarius    16
##  5 CORYBIF005 corydalis bifarius     1
##  6 CORYBIF006 corydalis bifarius    11
##  7 CORYBIF007 corydalis bifarius    10
##  8 CORYBIF008 corydalis bifarius    10
##  9 CORYBIF009 corydalis bifarius     1
## 10 CORYBIF010 corydalis bifarius     1
## # i 110 more rows
\end{verbatim}

\begin{Shaded}
\begin{Highlighting}[]
\NormalTok{robbed\_flowers }\SpecialCharTok{\%\textgreater{}\%} \FunctionTok{group\_by}\NormalTok{(sample, plantsp, species, bee) }\SpecialCharTok{\%\textgreater{}\%} \FunctionTok{count}\NormalTok{()}
\end{Highlighting}
\end{Shaded}

\begin{verbatim}
## # A tibble: 49 x 5
## # Groups:   sample, plantsp, species, bee [49]
##    sample     plantsp   species    bee     n
##    <chr>      <chr>     <chr>    <int> <int>
##  1 CORYBIF001 corydalis bifarius     3     3
##  2 CORYBIF004 corydalis bifarius     3    15
##  3 CORYBIF006 corydalis bifarius     7    11
##  4 CORYBIF007 corydalis bifarius     8    10
##  5 CORYBIF008 corydalis bifarius     2     9
##  6 CORYBIF011 corydalis bifarius     7    12
##  7 CORYBIF012 corydalis bifarius     8    19
##  8 CORYBIF014 corydalis bifarius     5    20
##  9 CORYBIF017 corydalis bifarius    11     9
## 10 CORYBIF019 corydalis bifarius     2     6
## # i 39 more rows
\end{verbatim}

Simple summary statistics for time to rob (mean, SE) broken down by
bumblebee species and plant sp. Note that time to rob is longer for
control flowers, but also SE higher for controls. Consistent response
for Mreu plants. Tactic switch, I just converted into a binary column
yes = 1, no = 0. Feeding time is higher in both treatments, with high SE
but consistent throughout.

\begin{Shaded}
\begin{Highlighting}[]
\NormalTok{summary\_stats }\OtherTok{\textless{}{-}}\NormalTok{ robbed\_flowers }\SpecialCharTok{\%\textgreater{}\%}
  \FunctionTok{group\_by}\NormalTok{(species, treatment, plantsp) }\SpecialCharTok{\%\textgreater{}\%}
  \FunctionTok{summarize}\NormalTok{(}
    \AttributeTok{Mean\_TimeToRob =} \FunctionTok{mean}\NormalTok{(timetorob, }\AttributeTok{na.rm =} \ConstantTok{TRUE}\NormalTok{),}
    \AttributeTok{SE\_TimeToRob =} \FunctionTok{sd}\NormalTok{(timetorob, }\AttributeTok{na.rm =} \ConstantTok{TRUE}\NormalTok{) }\SpecialCharTok{/} \FunctionTok{sqrt}\NormalTok{(}\FunctionTok{n}\NormalTok{()),}
    \AttributeTok{Mean\_TimeFeeding =} \FunctionTok{mean}\NormalTok{(timefeeding, }\AttributeTok{na.rm =} \ConstantTok{TRUE}\NormalTok{),}
    \AttributeTok{SE\_TimeFeeding =} \FunctionTok{sd}\NormalTok{(timetorob, }\AttributeTok{na.rm =} \ConstantTok{TRUE}\NormalTok{) }\SpecialCharTok{/} \FunctionTok{sqrt}\NormalTok{(}\FunctionTok{n}\NormalTok{()),}
    \AttributeTok{Mean\_tacticswitch =} \FunctionTok{mean}\NormalTok{(trytolegit\_binary, }\AttributeTok{na.rm =} \ConstantTok{TRUE}\NormalTok{),}
    \AttributeTok{SE\_tacticswitch =} \FunctionTok{sd}\NormalTok{(trytolegit\_binary, }\AttributeTok{na.rm =} \ConstantTok{TRUE}\NormalTok{) }\SpecialCharTok{/} \FunctionTok{sqrt}\NormalTok{(}\FunctionTok{n}\NormalTok{())}
\NormalTok{  ) }\SpecialCharTok{\%\textgreater{}\%} 
  \FunctionTok{arrange}\NormalTok{(plantsp, treatment)}
\end{Highlighting}
\end{Shaded}

\begin{verbatim}
## `summarise()` has grouped output by 'species', 'treatment'. You can override
## using the `.groups` argument.
\end{verbatim}

\begin{Shaded}
\begin{Highlighting}[]
\FunctionTok{print}\NormalTok{(summary\_stats)}
\end{Highlighting}
\end{Shaded}

\begin{verbatim}
## # A tibble: 6 x 9
## # Groups:   species, treatment [4]
##   species    treatment plantsp   Mean_TimeToRob SE_TimeToRob Mean_TimeFeeding
##   <chr>      <chr>     <chr>              <dbl>        <dbl>            <dbl>
## 1 bifarius   control   corydalis           5.42        0.637             7.76
## 2 flavifrons control   corydalis           5.24        0.672             8.04
## 3 bifarius   mreu      corydalis           4.16        0.343             7.09
## 4 flavifrons mreu      corydalis           3.20        0.272             8.04
## 5 flavifrons control   mertensia           7.12        1.08              9.25
## 6 flavifrons mreu      mertensia           3.97        0.426             8.06
## # i 3 more variables: SE_TimeFeeding <dbl>, Mean_tacticswitch <dbl>,
## #   SE_tacticswitch <dbl>
\end{verbatim}

\begin{Shaded}
\begin{Highlighting}[]
\FunctionTok{write.table}\NormalTok{(summary\_stats, }\StringTok{"results/summary\_feeding\_stats.csv"}\NormalTok{, }\AttributeTok{row.names =} \ConstantTok{FALSE}\NormalTok{)}
\end{Highlighting}
\end{Shaded}

\begin{Shaded}
\begin{Highlighting}[]
\CommentTok{\# Summarize the data to calculate the number of visits}
\NormalTok{summary\_data }\OtherTok{\textless{}{-}}\NormalTok{ robbed\_flowers }\SpecialCharTok{\%\textgreater{}\%}
  \FunctionTok{group\_by}\NormalTok{(treatment, species, plantsp) }\SpecialCharTok{\%\textgreater{}\%}
  \FunctionTok{summarize}\NormalTok{(}
    \AttributeTok{number\_of\_visits =} \FunctionTok{n}\NormalTok{(),  }\CommentTok{\# Count the number of visits}
    \AttributeTok{.groups =} \StringTok{"drop"}
\NormalTok{  )}

\CommentTok{\# Calculate the mean, stdev, and se for each treatment group (control vs mreu)}
\NormalTok{visit\_summary }\OtherTok{\textless{}{-}}\NormalTok{ summary\_data }\SpecialCharTok{\%\textgreater{}\%}
  \FunctionTok{group\_by}\NormalTok{(species, plantsp) }\SpecialCharTok{\%\textgreater{}\%}
  \FunctionTok{summarize}\NormalTok{(}
    \AttributeTok{mean\_visits\_control =} \FunctionTok{mean}\NormalTok{(number\_of\_visits[treatment }\SpecialCharTok{==} \StringTok{"control"}\NormalTok{], }\AttributeTok{na.rm =} \ConstantTok{TRUE}\NormalTok{),}
    \AttributeTok{mean\_visits\_treated =} \FunctionTok{mean}\NormalTok{(number\_of\_visits[treatment }\SpecialCharTok{==} \StringTok{"mreu"}\NormalTok{], }\AttributeTok{na.rm =} \ConstantTok{TRUE}\NormalTok{),}
    \AttributeTok{stdev\_diff =} \FunctionTok{sd}\NormalTok{(number\_of\_visits[treatment }\SpecialCharTok{==} \StringTok{"mreu"}\NormalTok{] }\SpecialCharTok{{-}}\NormalTok{ number\_of\_visits[treatment }\SpecialCharTok{==} \StringTok{"control"}\NormalTok{], }\AttributeTok{na.rm =} \ConstantTok{TRUE}\NormalTok{),}
    \AttributeTok{se\_diff =}\NormalTok{ stdev\_diff }\SpecialCharTok{/} \FunctionTok{sqrt}\NormalTok{(}\FunctionTok{n}\NormalTok{()),}
    \AttributeTok{.groups =} \StringTok{"drop"}
\NormalTok{  )}

\CommentTok{\# View the resulting summary table}
\FunctionTok{print}\NormalTok{(visit\_summary)}
\end{Highlighting}
\end{Shaded}

\begin{verbatim}
## # A tibble: 3 x 6
##   species    plantsp  mean_visits_control mean_visits_treated stdev_diff se_diff
##   <chr>      <chr>                  <dbl>               <dbl>      <dbl>   <dbl>
## 1 bifarius   corydal~                 114                 141         NA      NA
## 2 flavifrons corydal~                  47                 128         NA      NA
## 3 flavifrons mertens~                  39                  78         NA      NA
\end{verbatim}

\begin{Shaded}
\begin{Highlighting}[]
\CommentTok{\# Perform t{-}tests for each species and plantsp combination}
\NormalTok{t\_test\_results }\OtherTok{\textless{}{-}}\NormalTok{ robbed\_flowers }\SpecialCharTok{\%\textgreater{}\%}
  \FunctionTok{group\_by}\NormalTok{(species, plantsp) }\SpecialCharTok{\%\textgreater{}\%}
  \FunctionTok{summarize}\NormalTok{(}
    \AttributeTok{t\_test =} \FunctionTok{list}\NormalTok{(}
      \FunctionTok{t.test}\NormalTok{(}
        \AttributeTok{x =}\NormalTok{ treatment }\SpecialCharTok{==} \StringTok{"mreu"}\NormalTok{,    }\CommentTok{\# Logical vector for mreu}
        \AttributeTok{y =}\NormalTok{ treatment }\SpecialCharTok{==} \StringTok{"control"} \CommentTok{\# Logical vector for control}
\NormalTok{      )}
\NormalTok{    ),}
    \AttributeTok{.groups =} \StringTok{"drop"}
\NormalTok{  )}

\CommentTok{\# Extract p{-}values and t{-}statistics from the t{-}test results}
\NormalTok{t\_test\_summary }\OtherTok{\textless{}{-}}\NormalTok{ t\_test\_results }\SpecialCharTok{\%\textgreater{}\%}
  \FunctionTok{mutate}\NormalTok{(}
    \AttributeTok{p\_value =} \FunctionTok{sapply}\NormalTok{(t\_test, }\ControlFlowTok{function}\NormalTok{(x) x}\SpecialCharTok{$}\NormalTok{p.value),        }\CommentTok{\# Extract p{-}value}
    \AttributeTok{statistic =} \FunctionTok{sapply}\NormalTok{(t\_test, }\ControlFlowTok{function}\NormalTok{(x) x}\SpecialCharTok{$}\NormalTok{statistic)    }\CommentTok{\# Extract t{-}statistic}
\NormalTok{  )}

\CommentTok{\# Select only necessary columns}
\NormalTok{t\_test\_summary }\OtherTok{\textless{}{-}}\NormalTok{ t\_test\_summary }\SpecialCharTok{\%\textgreater{}\%}
\NormalTok{  dplyr}\SpecialCharTok{::}\FunctionTok{select}\NormalTok{(species, plantsp, p\_value, statistic)}

\CommentTok{\# View the summarized results}
\FunctionTok{print}\NormalTok{(t\_test\_summary)}
\end{Highlighting}
\end{Shaded}

\begin{verbatim}
## # A tibble: 3 x 4
##   species    plantsp    p_value statistic
##   <chr>      <chr>        <dbl>     <dbl>
## 1 bifarius   corydalis 1.68e- 2      2.40
## 2 flavifrons corydalis 5.52e-20      9.74
## 3 flavifrons mertensia 1.77e- 7      5.39
\end{verbatim}

\begin{Shaded}
\begin{Highlighting}[]
\CommentTok{\# Extract p{-}values, t{-}statistics, and degrees of freedom from the t{-}test results}
\NormalTok{t\_test\_summary }\OtherTok{\textless{}{-}}\NormalTok{ t\_test\_results }\SpecialCharTok{\%\textgreater{}\%}
  \FunctionTok{mutate}\NormalTok{(}
    \AttributeTok{p\_value =} \FunctionTok{sapply}\NormalTok{(t\_test, }\ControlFlowTok{function}\NormalTok{(x) x}\SpecialCharTok{$}\NormalTok{p.value),        }\CommentTok{\# Extract p{-}value}
    \AttributeTok{statistic =} \FunctionTok{sapply}\NormalTok{(t\_test, }\ControlFlowTok{function}\NormalTok{(x) x}\SpecialCharTok{$}\NormalTok{statistic),    }\CommentTok{\# Extract t{-}statistic}
    \AttributeTok{df =} \FunctionTok{sapply}\NormalTok{(t\_test, }\ControlFlowTok{function}\NormalTok{(x) x}\SpecialCharTok{$}\NormalTok{parameter)            }\CommentTok{\# Extract degrees of freedom}
\NormalTok{  )}

\CommentTok{\# Explicitly use dplyr::select()}
\NormalTok{t\_test\_summary }\OtherTok{\textless{}{-}}\NormalTok{ t\_test\_summary }\SpecialCharTok{\%\textgreater{}\%}
\NormalTok{  dplyr}\SpecialCharTok{::}\FunctionTok{select}\NormalTok{(species, plantsp, p\_value, statistic, df)}

\CommentTok{\# View the summarized results}
\FunctionTok{print}\NormalTok{(t\_test\_summary)}
\end{Highlighting}
\end{Shaded}

\begin{verbatim}
## # A tibble: 3 x 5
##   species    plantsp    p_value statistic    df
##   <chr>      <chr>        <dbl>     <dbl> <dbl>
## 1 bifarius   corydalis 1.68e- 2      2.40   508
## 2 flavifrons corydalis 5.52e-20      9.74   348
## 3 flavifrons mertensia 1.77e- 7      5.39   232
\end{verbatim}

\begin{Shaded}
\begin{Highlighting}[]
\CommentTok{\# Create a histogram}
\FunctionTok{ggplot}\NormalTok{(robbed\_flowers, }\FunctionTok{aes}\NormalTok{(}\AttributeTok{x =} \FunctionTok{interaction}\NormalTok{(species, plantsp), }\AttributeTok{fill =}\NormalTok{ treatment)) }\SpecialCharTok{+}
  \FunctionTok{geom\_bar}\NormalTok{(}\AttributeTok{position =} \StringTok{"dodge"}\NormalTok{, }\AttributeTok{color =} \StringTok{"black"}\NormalTok{) }\SpecialCharTok{+}
  \FunctionTok{labs}\NormalTok{(}\AttributeTok{x =} \ConstantTok{NULL}\NormalTok{,}
    \AttributeTok{y =} \StringTok{"Number of Visits"}\NormalTok{,}
    \AttributeTok{fill =} \StringTok{"Flower Treatment"}
\NormalTok{  ) }\SpecialCharTok{+}
  \FunctionTok{scale\_fill\_manual}\NormalTok{(}
    \AttributeTok{values =} \FunctionTok{c}\NormalTok{(}\StringTok{"mreu"} \OtherTok{=} \StringTok{"\#4169E1"}\NormalTok{,}
               \StringTok{"control"} \OtherTok{=} \StringTok{"\#FAFAD2"}\NormalTok{)}
\NormalTok{  )}\SpecialCharTok{+}
  \FunctionTok{theme\_minimal}\NormalTok{() }\SpecialCharTok{+}
  \FunctionTok{theme}\NormalTok{(}
    \AttributeTok{legend.title =} \FunctionTok{element\_text}\NormalTok{(}\AttributeTok{size =} \DecValTok{10}\NormalTok{),}
    \AttributeTok{legend.text =} \FunctionTok{element\_text}\NormalTok{(}\AttributeTok{size =} \DecValTok{9}\NormalTok{)}
\NormalTok{  ) }\OtherTok{{-}\textgreater{}}\NormalTok{ plot}

\NormalTok{plot}
\end{Highlighting}
\end{Shaded}

\pandocbounded{\includegraphics[keepaspectratio]{bumblebeeRobbing_script2_files/figure-latex/unnamed-chunk-2-1.pdf}}

\begin{Shaded}
\begin{Highlighting}[]
\FunctionTok{ggsave}\NormalTok{(}\StringTok{"results/bee\_visits\_histogram.pdf"}\NormalTok{, }\AttributeTok{plot =}\NormalTok{ plot, }\AttributeTok{width =} \DecValTok{8}\NormalTok{, }\AttributeTok{height =} \DecValTok{6}\NormalTok{)}
\end{Highlighting}
\end{Shaded}

\begin{Shaded}
\begin{Highlighting}[]
\NormalTok{summary\_df }\OtherTok{\textless{}{-}}\NormalTok{ robbed\_flowers }\SpecialCharTok{\%\textgreater{}\%}
  \FunctionTok{group\_by}\NormalTok{(}\AttributeTok{group =} \FunctionTok{interaction}\NormalTok{(species, plantsp), treatment) }\SpecialCharTok{\%\textgreater{}\%}
  \FunctionTok{summarise}\NormalTok{(}\AttributeTok{count =} \FunctionTok{n}\NormalTok{(), }\AttributeTok{.groups =} \StringTok{"drop"}\NormalTok{) }\SpecialCharTok{\%\textgreater{}\%}
  \FunctionTok{group\_by}\NormalTok{(group) }\SpecialCharTok{\%\textgreater{}\%}
  \FunctionTok{summarise}\NormalTok{(}
    \AttributeTok{max\_y =} \FunctionTok{max}\NormalTok{(count),}
    \AttributeTok{x\_pos =} \FunctionTok{as.numeric}\NormalTok{(}\FunctionTok{factor}\NormalTok{(group)),  }\CommentTok{\# for proper placement on x axis}
    \AttributeTok{.groups =} \StringTok{"drop"}
\NormalTok{  ) }\SpecialCharTok{\%\textgreater{}\%}
  \FunctionTok{mutate}\NormalTok{(}\AttributeTok{label =} \StringTok{"*"}\NormalTok{)}
\end{Highlighting}
\end{Shaded}

\begin{verbatim}
## Warning: Returning more (or less) than 1 row per `summarise()` group was deprecated in
## dplyr 1.1.0.
## i Please use `reframe()` instead.
## i When switching from `summarise()` to `reframe()`, remember that `reframe()`
##   always returns an ungrouped data frame and adjust accordingly.
## Call `lifecycle::last_lifecycle_warnings()` to see where this warning was
## generated.
\end{verbatim}

\begin{Shaded}
\begin{Highlighting}[]
\FunctionTok{ggplot}\NormalTok{(robbed\_flowers, }\FunctionTok{aes}\NormalTok{(}\AttributeTok{x =} \FunctionTok{interaction}\NormalTok{(species, plantsp), }\AttributeTok{fill =}\NormalTok{ treatment)) }\SpecialCharTok{+}
  \FunctionTok{geom\_bar}\NormalTok{(}\AttributeTok{position =} \FunctionTok{position\_dodge}\NormalTok{(}\AttributeTok{width =} \FloatTok{0.9}\NormalTok{), }\AttributeTok{color =} \StringTok{"black"}\NormalTok{) }\SpecialCharTok{+}
  \FunctionTok{geom\_text}\NormalTok{(}
    \AttributeTok{data =}\NormalTok{ summary\_df,}
    \FunctionTok{aes}\NormalTok{(}\AttributeTok{x =}\NormalTok{ group, }\AttributeTok{y =}\NormalTok{ max\_y }\SpecialCharTok{+} \DecValTok{2}\NormalTok{, }\AttributeTok{label =}\NormalTok{ label),}
    \AttributeTok{inherit.aes =} \ConstantTok{FALSE}\NormalTok{,}
    \AttributeTok{size =} \DecValTok{6}
\NormalTok{  ) }\SpecialCharTok{+}
  \FunctionTok{labs}\NormalTok{(}\AttributeTok{x =} \ConstantTok{NULL}\NormalTok{,}
       \AttributeTok{y =} \StringTok{"Number of Visits"}\NormalTok{,}
       \AttributeTok{fill =} \StringTok{"Flower Treatment"}\NormalTok{) }\SpecialCharTok{+}
  \FunctionTok{scale\_fill\_manual}\NormalTok{(}
    \AttributeTok{values =} \FunctionTok{c}\NormalTok{(}\StringTok{"mreu"} \OtherTok{=} \StringTok{"\#4169E1"}\NormalTok{, }\StringTok{"control"} \OtherTok{=} \StringTok{"\#FAFAD2"}\NormalTok{)}
\NormalTok{  ) }\SpecialCharTok{+}
  \FunctionTok{theme\_minimal}\NormalTok{() }\SpecialCharTok{+}
  \FunctionTok{theme}\NormalTok{(}
    \AttributeTok{legend.title =} \FunctionTok{element\_text}\NormalTok{(}\AttributeTok{size =} \DecValTok{10}\NormalTok{),}
    \AttributeTok{legend.text =} \FunctionTok{element\_text}\NormalTok{(}\AttributeTok{size =} \DecValTok{9}\NormalTok{)}
\NormalTok{  ) }\OtherTok{{-}\textgreater{}}\NormalTok{ plot2}

\NormalTok{plot2}
\end{Highlighting}
\end{Shaded}

\pandocbounded{\includegraphics[keepaspectratio]{bumblebeeRobbing_script2_files/figure-latex/unnamed-chunk-3-1.pdf}}

\end{document}
